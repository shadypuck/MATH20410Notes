\documentclass[../psets.tex]{subfiles}

\pagestyle{main}
\renewcommand{\leftmark}{Problem Set \thesection}
\setcounter{section}{6}

\begin{document}




\section{Functions of Several Variables III}
\emph{From \textcite{bib:Rudin}.}
\subsection*{Chapter 9}
\begin{enumerate}[label={\textbf{\arabic*.}}]
    \setcounter{enumi}{8}
    \item \marginnote{3/2:}If $\mathbf{f}$ is a differentiable mapping of a \emph{connected} open set $E\subset\R^n$ into $\R^m$, and if $\mathbf{f}'(\vec{x})=0$ for every $\vec{x}\in E$, prove that $\mathbf{f}$ is constant in $E$.
    \setcounter{enumi}{12}
    \item Suppose $\mathbf{f}$ is a differentiable mapping of $\R^1$ into $\R^3$ such that $|\mathbf{f}(t)|=1$ for every $t$. Prove that $\mathbf{f}'(t)\cdot\mathbf{f}(t)=0$. Interpret this result geometrically.
    \item Define
    \begin{equation*}
        f(x,y) =
        \begin{cases}
            0 & (x,y)=(0,0)\\
            \frac{x^3}{x^2+y^2} & (x,y)\neq(0,0)
        \end{cases}
    \end{equation*}
    \begin{enumerate}
        \item Prove that $D_1f$ and $D_2f$ are bounded functions in $\R^2$. (Hence $f$ is continuous.)
        \item Let $\mathbf{u}$ be any unit vector in $\R^2$. Show that the directional derivative $(D_\vec{u}f)(0,0)$ exists, and that its absolute value is at most 1.
        \item Let $\gamma$ be a differentiable mapping of $\R^1$ into $\R^2$ (in other words, $\gamma$ is a differentiable curve in $\R^2$), with $\gamma(0)=(0,0)$ and $|\gamma'(0)|>0$. Put $g(t)=f(\gamma(t))$ and prove that $g$ is differentiable for every $t\in\R^1$. If $\gamma\in C^1$, prove that $g\in C^1$.
        \item In spite of this, prove that $f$ is not differentiable at $(0,0)$. (Hint: The formula $(D_\vec{u}f)(\vec{x})=\sum_{i=1}^n(D_if)(\vec{x})u_i$ fails.)
    \end{enumerate}
    \stepcounter{enumi}
    \item Show that the continuity of $\mathbf{f}'$ at the point $\vec{a}$ is needed in the inverse function theorem, even in the case $n=1$: If
    \begin{equation*}
        f(t) =
        \begin{cases}
            t+2t^2\sin\frac{1}{t} & t\neq 0\\
            0 & t=0
        \end{cases}
    \end{equation*}
    then $f'(0)=1$ and $f'$ is bounded in $(-1,1)$, but $f$ is not one-to-one in any neighborhood of 0.
\end{enumerate}




\end{document}