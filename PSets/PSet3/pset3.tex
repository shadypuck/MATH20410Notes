\documentclass[../psets.tex]{subfiles}

\pagestyle{main}
\renewcommand{\leftmark}{Problem Set \thesection}
\setcounter{section}{2}

\begin{document}




\section{Integration II}
\emph{From \textcite{bib:Rudin}.}
\subsection*{Chapter 6}
\begin{enumerate}[label={\textbf{\arabic*.}}]
    \setcounter{enumi}{2}
    \item \marginnote{2/2:}Define three functions $\beta_1,\beta_2,\beta_3$ as follows:
    \begin{align*}
        \beta_1 &=
        \begin{cases}
            0 & x<0\\
            0 & x=0\\
            1 & x>0
        \end{cases}&
        \beta_2 &=
        \begin{cases}
            0 & x<0\\
            1 & x=0\\
            1 & x>0
        \end{cases}&
        \beta_3 &=
        \begin{cases}
            0 & x<0\\
            \tfrac{1}{2} & x=0\\
            1 & x>0
        \end{cases}
    \end{align*}
    Let $f$ be a bounded function on $[-1,1]$.
    \begin{enumerate}
        \item Prove that $f\in\mathscr{R}(\beta_1)$ if and only if $f(0+)=f(0)$ and that then
        \begin{equation*}
            \int f\dd{\beta_1} = f(0)
        \end{equation*}
        \item State and prove a similar result for $\beta_2$.
        \item Prove that $f\in\mathscr{R}(\beta_3)$ if and only if $f$ is continuous at 0.
        \item If $f$ is continuous at 0, prove that
        \begin{equation*}
            \int f\dd{\beta_1} = \int f\dd{\beta_2}
            = \int f\dd{\beta_3}
            = f(0)
        \end{equation*}
    \end{enumerate}
    \stepcounter{enumi}
    \item Suppose $f$ is a bounded real function on $[a,b]$, and $f^2\in\mathscr{R}$ on $[a,b]$. Does it follow that $f\in\mathscr{R}$? Does the answer change if we assume that $f^3\in\mathscr{R}$?
    \stepcounter{enumi}
    \item Suppose $f$ is a real function on $(0,1]$ and $f\in\mathscr{R}$ on $[c,1]$ for every $c>0$. Define
    \begin{equation*}
        \int_0^1f(x)\dd{x} = \lim_{c\to 0}\int_c^1f(x)\dd{x}
    \end{equation*}
    if this limit exists (and is finite).
    \begin{enumerate}
        \item If $f\in\mathscr{R}$ on $[0,1]$, show that this definition of the integral agrees with the old one.
        \item Construct a function $f$ such that the above limit exists, although it fails to exist with $|f|$ in place of $f$.
    \end{enumerate}
    \item Suppose $f\in\mathscr{R}$ on $[a,b]$ for every $b>a$ where $a$ is fixed. Define
    \begin{equation*}
        \int_a^\infty f(x)\dd{x} = \lim_{b\to\infty}\int_a^bf(x)\dd{x}
    \end{equation*}
    if this limit exists (and is finite). In that case, we say that the integral on the left \textbf{converges}. If it also converges after $f$ has been replaced by $|f|$, it is said to converge \textbf{absolutely}.\par
    Assume that $f(x)\geq 0$ and that $f$ decreases monotonically on $[1,\infty)$. Prove that $\int_1^\infty f(x)\dd{x}$ converges if and only if $\sum_{n=1}^\infty f(n)$ converges. (This is the so-called "integral test" for convergence of series.)
    \stepcounter{enumi}
    \item Let $p,q$ be positive real numbers such that
    \begin{equation*}
        \frac{1}{p}+\frac{1}{q} = 1
    \end{equation*}
    Prove the following statements.
    \begin{enumerate}
        \item If $u,v\geq 0$, then
        \begin{equation*}
            uv \leq \frac{u^p}{p}+\frac{v^q}{q}
        \end{equation*}
        Equality holds if and only if $u^p=v^q$.
        \item If $f,g\in\mathscr{R}(\alpha)$, $f,g\geq 0$, and
        \begin{equation*}
            \int_a^bf^p\dd{\alpha} = 1 = \int_a^bg^q\dd{\alpha}
        \end{equation*}
        then
        \begin{equation*}
            \int_a^b fg\dd{\alpha} \leq 1
        \end{equation*}
        \item If $f,g$ are complex functions in $\mathscr{R}(\alpha)$, then
        \begin{equation*}
            \left| \int_a^bfg\dd{\alpha} \right| \leq \left( \int_a^b|f|^p\dd{\alpha} \right)^{1/p}\left( \int_a^b|g|^q\dd{\alpha} \right)^{1/q}
        \end{equation*}
        This is \textbf{H\"{o}lder's inequality}. When $p=q=2$, it is usually called the Schwarz inequality. (Note that Theorem 1.35 is a very special case of this.)
    \end{enumerate}
    \item Let $\alpha$ be a fixed increasing function on $[a,b]$. For $u\in\mathscr{R}(\alpha)$, define
    \begin{equation*}
        \norm{u}_2 = \left( \int_a^b|u|^2\dd{\alpha} \right)^{1/2}
    \end{equation*}
    Suppose $f,g,h\in\mathscr{R}(\alpha)$, and prove the triangle inequality
    \begin{equation*}
        \norm{f-h}_2 \leq \norm{f-g}_2+\norm{g-h}_2
    \end{equation*}
    as a consequence of the Schwarz inequality, as in the proof of Theorem 1.37.
\end{enumerate}




\end{document}