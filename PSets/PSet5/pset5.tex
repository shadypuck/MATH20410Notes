\documentclass[../psets.tex]{subfiles}

\pagestyle{main}
\renewcommand{\leftmark}{Problem Set \thesection}
\setcounter{section}{3}

\begin{document}




\section{Sequences and Series of Functions II / Functions of Several Variables}
\emph{From \textcite{bib:Rudin}.}
\subsection*{Chapter 7}
\begin{enumerate}[label={\textbf{\arabic*.}}]
    \setcounter{enumi}{4}
    \item \marginnote{2/16:}Let
    \begin{equation*}
        f_n(x) =
        \begin{cases}
            0 & x<\frac{1}{n+1}\\
            \sin^2\frac{\pi}{x} & \frac{1}{n+1}\leq x\leq\frac{1}{n}\\
            0 & \frac{1}{n}<x
        \end{cases}
    \end{equation*}
    Show that $\{f_n\}$ converges to a continuous function, but not uniformly. Use the series $\sum f_n$ to show that absolute convergence, even for all $x$, does not imply uniform convergence.
    \item Prove that the series
    \begin{equation*}
        \sum_{n=1}^\infty(-1)^n\frac{x^2+n}{n^2}
    \end{equation*}
    converges uniformly in every bounded interval, but does not converge absolutely for any value of $x$.
    \stepcounter{enumi}
    \item If
    \begin{equation*}
        I(x) =
        \begin{cases}
            0 & x\leq 0\\
            1 & x>0
        \end{cases}
    \end{equation*}
    if $\{x_n\}$ is a sequence of distinct points of $(a,b)$, and if $\sum|c_n|$ converges, prove that the series
    \begin{equation*}
        f(x) = \sum_{n=1}^\infty c_nI(x-x_n)
    \end{equation*}
    converges uniformly on $[a,b]$, and that $f$ is continuous for every $x\neq x_n$.
    \item Let $\{f_n\}$ be a sequence of continuous functions which converges uniformly to a function $f$ on a set $E$. Prove that
    \begin{equation*}
        \lim_{n\to\infty}f_n(x_n) = f(x)
    \end{equation*}
    for every sequence of points $x_n\in E$ such that $x_n\to x$ and $x\in E$. Is the converse of this true?
\end{enumerate}


\subsection*{Chapter 9}
\begin{enumerate}[label={\textbf{\arabic*.}}]
    \item If $S$ is a nonempty subset of a vector space $X$, prove (as asserted in Section 9.1) that the span of $S$ is a vector space.
    \item Prove (as asserted in Section 9.6) that $BA$ is linear if $A$ and $B$ are linear transformations. Prove also that $A^{-1}$ is linear and invertible.
    \item Assume $A\in L(X,Y)$ and $A\vec{x}=\bm{0}$ only when $\vec{x}=\bm{0}$. Prove that $A$ is then 1-1.
    \item Prove (as asserted in Section 9.30) that null spaces and ranges of linear transformations are vector spaces.
\end{enumerate}




\end{document}