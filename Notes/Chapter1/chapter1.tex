\documentclass[../notes.tex]{subfiles}

\pagestyle{main}
\renewcommand{\chaptermark}[1]{\markboth{Chapter 1 (The Algebra and Topology of $\R^n$)}{}}

\begin{document}




\chapter{The Algebra and Topology of \texorpdfstring{$\pmb{\R}^{\bm{n}}$}{TEXT}}
\section{Notes}
\begin{itemize}
    \item \marginnote{1/10:}Syllabus.
    \begin{itemize}
        \item In his mind, homework is the main setting where learning takes place.
    \end{itemize}
    \item We're going to be studying analysis, or calculus, on \textbf{manifolds} this quarter.
    \item \textbf{Manifold}: A "space" that looks like Euclidean space $\R^n$ locally.
    \begin{itemize}
        \item The surfaces of a sphere and torus are common examples of 2-dimensional manifolds.
        \item With regard to the above definition, think about how people in ancient times didn't think the Earth was a sphere because it looked like a plane locally.
    \end{itemize}
    \item This class will look much like a calculus course, in that we first talk about limits, then differentiation, then integration, and culminating in the fundamental theory of calculus.
    \item Last quarter, we primarily developed linear algebra and basic topology on metric spaces.
    \begin{itemize}
        \item Chapter 1 of \textcite{bib:Munkres} is a review of what's needed from last quarter.
        \item This is all basically continuity.
    \end{itemize}
    \item Thus, we can start right up with differentiation.
\end{itemize}




\end{document}