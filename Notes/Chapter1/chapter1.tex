\documentclass[../notes.tex]{subfiles}

\pagestyle{main}
\renewcommand{\chaptermark}[1]{\markboth{Chapter 1 (The Algebra and Topology of $\R^n$)}{}}

\begin{document}




\chapter{The Algebra and Topology of \texorpdfstring{$\pmb{\R}^{\bm{n}}$}{TEXT}}
\section{Notes}
\begin{itemize}
    \item \marginnote{1/10:}Syllabus.
    \begin{itemize}
        \item In his mind, homework is the main setting where learning takes place.
    \end{itemize}
    \item We're going to be studying analysis, or calculus, on \textbf{manifolds} this quarter.
    \item \textbf{Manifold}: A "space" that looks like Euclidean space $\R^n$ locally.
    \begin{itemize}
        \item The surfaces of a sphere and torus are common examples of 2-dimensional manifolds.
        \item With regard to the above definition, think about how people in ancient times didn't think the Earth was a sphere because it looked like a plane locally.
    \end{itemize}
    \item This class will look much like a calculus course, in that we first talk about limits, then differentiation, then integration, and culminating in the fundamental theory of calculus.
    \item Last quarter, we primarily developed linear algebra and basic topology on metric spaces.
    \begin{itemize}
        \item Chapter 1 of \textcite{bib:Munkres} is a review of what's needed from last quarter.
        \item This is all basically continuity.
    \end{itemize}
    \item Thus, we can start right up with differentiation.
\end{itemize}



\section[The Algebra and Topology of \texorpdfstring{$\R^n$}{TEXT}]{The Algebra and Topology of \texorpdfstring{$\pmb{\R}^{\bm{n}}$}{TEXT}}
\emph{From \textcite{bib:Munkres}.}
\begin{itemize}
    \item \marginnote{1/17:}"In the first part of this book, $\R^n$ and its subspaces are the only vector spaces with which we shall be concerned. In later chapters, we shall deal with more general vector spaces" \parencite[2]{bib:Munkres}.
    \item \textbf{Inner product}: \emph{Denoted by} $\bm{\langle\vec{x},\vec{y}\rangle}$.
    \item \textbf{Euclidean norm}: The following norm. \emph{Denoted by} $\pmb{\parallel}\!\vec{x}\!\pmb{\parallel}$. \emph{Given by}
    \begin{equation*}
        \norm{\vec{x}} = \sqrt{x_1^2+\cdots+x_n^2}
    \end{equation*}
    \item \textbf{Sup norm} (of $n$-tuples): The following norm. \emph{Denoted by} $\pmb{|}\vec{x}\pmb{|}$. \emph{Given by}
    \begin{equation*}
        |\vec{x}| = \max\{|x_1|,\dots,|x_n|\}
    \end{equation*}
    \item Note that the Euclidean norm and sup norm satisfy the inequalities
    \begin{equation*}
        |\vec{x}| \leq \norm{\vec{x}} \leq \sqrt{n}|\vec{x}|
    \end{equation*}
    for all $\vec{x}\in\R^n$.
    \item \textbf{Sup norm} (of matrices): The following norm. \emph{Denoted by} $\pmb{|}\bm{A}\pmb{|}$. \emph{Given by}
    \begin{equation*}
        |A| = \max\{|a_{ij}|:i\in[n],\ j\in[m]\}
    \end{equation*}
    \item Theorem 1.3: If $A$ has size $n$ by $m$ and $B$ has size $m$ by $p$, then
    \begin{equation*}
        |A\cdot B| \leq m|A|\,|B|
    \end{equation*}
    \item \textbf{Echelon form}: \emph{Also known as} \textbf{stairstep form}.
    \item \textbf{Transpose} (of $A$): \emph{Denoted by} $\bm{A^\textbf{tr}}$.
    \item Theorem 2.1: Let $A$ be an $n$-by-$m$ matrix. Any elementary row operation on $A$ may be carried out by premultiplying $A$ by the corresponding elementary matrix.
    \begin{itemize}
        \item "We will use this result later on when we prove the change of variables theorem for a multiple integral" \parencite[12]{bib:Munkres}.
    \end{itemize}
    \item \textbf{Determinant} (of $A$): \emph{Denoted by} $\textbf{det}\,\bm{A}$. \emph{Not denoted by} $|A|$.
    \item \textbf{Determinant function}: A function that assigns to each $n$-by-$n$ matrix $A$ a real number denoted $\det A$ and satisfies the following axioms.
    \begin{enumerate}
        \item If $B$ is the matrix obtained by exchanging any two rows of $A$, then $\det B=-\det A$.
        \item Given $i$, the function $\det A$ is linear as a function of the $i^\text{th}$ row alone.
        \item $\det I_n=1$.
    \end{enumerate}
    \item Corollary 2.9: The determinant function is uniquely characterized by its three axioms.
    \item \textbf{$\bm{\epsilon}$-neighborhood} (of $x_0$): The following set, where $X$ is a metric space with metric $d$, $x_0\in X$, and $\epsilon>0$. \emph{Also known as} \textbf{$\bm{\epsilon}$-neighborhood centered at $\bm{x_0}$}. \emph{Denoted by} $\bm{U(x_0;\epsilon)}$. \emph{Given by}
    \begin{equation*}
        U(x_0;\epsilon) = \{x\mid d(x,x_0)<\epsilon\}
    \end{equation*}
    \item \textbf{Topological property} (of $X$): A property of a metric space $X$ that depends only on the collection of open sets of $X$, rather than on the specific metric involved.
    \begin{itemize}
        \item Examples include limits, continuity, and compactness.
    \end{itemize}
    \item \textbf{Interior} (of $A\subset\R^n$): The union of all open sets of $\R^n$ that are contained in $A$. \emph{Denoted by} $\textbf{Int}\,\bm{A}$.
    \item \textbf{Exterior} (of $A\subset\R^n$): The union of all open sets of $\R^n$ that are disjoint from $A$. \emph{Denoted by} $\textbf{Ext}\,\bm{A}$.
    \item \textbf{Boundary} (of $A\subset\R^n$): The set of all points of $\R^n$ that are contained in neither $\Int A$ nor $\Ext A$. \emph{Denoted by} $\textbf{Bd}\,\bm{A}$.
    \begin{itemize}
        \item $\vec{x}\in\Bd A$ iff every open set containing $\vec{x}$ intersects both $A$ and $\R^n\setminus A$.
    \end{itemize}
\end{itemize}




\end{document}