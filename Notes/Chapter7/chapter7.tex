\documentclass[../notes.tex]{subfiles}

\pagestyle{main}
\renewcommand{\chaptermark}[1]{\markboth{\chaptername\ \thechapter\ (#1)}{}}
\setcounter{chapter}{6}

\begin{document}




\chapter{Sequences and Series of Functions}
\section{Notes}
\begin{itemize}
    \item \marginnote{1/31:}Midterm on differentiation and integration, and a bit of stuff from this week.
    \item Plan:
    \begin{itemize}
        \item Talk about sequences of functions, all with the same domain and range, converging.
        \item Address what properties of $f_n$ remain in the limit (e.g., continuity, differentiability, integrability).
        \begin{itemize}
            \item The answer depends on what we mean by "convergence."
            \item $f_n\to f$ pointwise implies basically nothing.
            \item $f_n\to f$ uniformly implies that basically everything works out nicely.
        \end{itemize}
    \end{itemize}
    \item We'll restrict ourselves to real functions because those have all the properties (integrability, differentiability, etc.) that we care about.
    \item \textbf{Pointwise} (convergent sequence $\{f_n\}$ to $f$): A sequence of functions $\{f_n\}$ such that for all $x\in X$, the sequence $\{f_n(x)\}$ converges to $f(x)$, where $f_n:X\to\R$ for all $n\in\N$ and $f:X\to\R$. \emph{Denoted by} $f_n\to f$.
    \item Bad functions.
    \begin{itemize}
        \item Consider $f_n:[0,1]\to\R$ defined by $x\mapsto x^n$. Each $f_n$ is continuous, but $f$ is not (zero everywhere except $f(1)=1$)\footnote{Questions that require counterexamples like this could show up on the midterm!}.
        \item Consider $f_n:\R\to\R$ defined by $f_n(x)=x^2/(1+x^2)^n$, and $f(x)=\sum_{n=0}^\infty f_n(x)$. As a geometric series, $f(x)=1+x^2$ when $x\neq 0$ but $f(0)=0$. Thus, the limit exists but is not continuous once again.
        \item Consider $f_m:\R\to\R$ defined by $x\mapsto\lim_{n\to\infty}\cos^{2n}(m!\pi x)$. Each $f_m$ is integrable, but the limit $f$ is the function that's 1 for rationals and zero for irrationals. In particular, $f$ is not integrable.
        \begin{itemize}
            \item We take even powers of the cosine to make it always positive.
            \item We use $\cos^2(x)$ just because its always between $[0,1]$, and we know when it is equal to 1.
            \item In particular, $\cos^2(\pi x)$ is equal to 1 at every integer, $\cos^2(2\pi x)$ is equal to 1 at every half integer. $\cos^2(6\pi x)$ is equal to 1 at every one-sixth of an integer.
            \item Then raising it to the $n^\text{th}$ power just makes it spiky.
        \end{itemize}
    \end{itemize}
    \item Aside: Interchanging limits.
    \begin{itemize}
        \item If all $f_n$ are continuous, then $\lim_{x\to x_0}f_n(x)=f_n(x_0)$.
        \item The question "is $f$ continuous" is equivalent to being able to interchange limits:
        \begin{equation*}
            \lim_{x\to x_0}\lim_{n\to\infty}f_n(x) = f(x_0) = \lim_{n\to\infty}\lim_{x\to x_0}f_n(x)
        \end{equation*}
        \item Sequence example showing we need to be careful interchanging limits: $s_{n,m}=m/(m+n)$.
    \end{itemize}
    \item All of this pathology goes away with the right definition, though.
    \item \textbf{Uniformly} (convergent sequence $\{f_n\}$ to $f$): A sequence of functions $\{f_n\}$ such that for all $\epsilon>0$, there exists an $N$ such that if $n\geq N$, then $|f_n(x)-f(x)|<\epsilon$ for all $x\in X$, where $f_n:X\to\R$ for all $n\in\N$ and $f:X\to\R$.
    \item Proposition (Cauchy criterion for uniform convergence): $f_n\to f$ uniformly iff for all $\epsilon>0$, there exists $N$ such that for all $m,n\geq N$ and for all $x\in X$, $|f_n(x)-f_m(x)|<\epsilon$.
    \begin{itemize}
        \item Forward direction: Let $\epsilon>0$. Suppose $f_n\to f$ uniformly. Choose $N$ such that the functions are within $\epsilon/2$. Then
        \begin{equation*}
            |f_n(x)-f_m(x)| \leq |f_n(x)-f(x)|+|f(x)-f_m(x)| < \frac{\epsilon}{2}+\frac{\epsilon}{2} = \epsilon
        \end{equation*}
    \end{itemize}
    \item \marginnote{2/2:}Office hours tomorrow 4-5 PM.
    \item Plan:
    \begin{enumerate}
        \item More on uniform convergence.
        \begin{itemize}
            \item Limit of continuous functions is continuous.
            \item Limit of the integral of functions is the integral of the limit.
        \end{itemize}
        \item $\mathcal{C}(X)$ perspectives on uniform convergence.
    \end{enumerate}
    \item Corollary (Weierstra{\ss} M-test): If there exist constants $M_n\in\R$ such that $|f_n(x)|\leq M_n$ for all $x$ and $\sum_{n=1}^\infty M_n$ converges, then $\sum_{n=1}^\infty f_n(x)$ converges uniformly.
    \item Theorem: $f_n:X\to\R$, $f_n$ continuous at $x_0\in X$ for all $n$, and $f_n\to f$ uniformly imply $f$ continuous at $x_0$.
    \begin{itemize}
        \item Idea:
        \begin{itemize}
            \item "$\epsilon/3$ trick": Find $\delta$ such that if $|x-x_0|<\delta$, then
            \begin{equation*}
                |f(x)-f(x_0)| \leq |f(x)-f_N(x)|+|f_N(x)-f_N(x_0)|+|f_N(x_0)-f(x_0)| < \frac{\epsilon}{3}+\frac{\epsilon}{3}+\frac{\epsilon}{3} = \epsilon
            \end{equation*}
        \end{itemize}
        \item Proof:
        \begin{itemize}
            \item $f_n\to f$ uniformly implies there exists $N\in\N$ such that $|f_N(x)-f(x)|<\epsilon/3$ for all $x\in X$.
            \item $f_N$ continuous at $x_0$: There exists $\delta$ such that if $d(x,x_0)<\delta$, then $|f_N(x)-f_N(x_0)|<\epsilon/3$.
            \item Thus, by the $\epsilon/3$ trick, we have the continuity of $f$.
        \end{itemize}
    \end{itemize}
    \item Defining a norm on $\mathcal{C}(X)$.
    \begin{equation*}
        \norm{f} = \sup_{x\in X}|f(x)|
    \end{equation*}
    \begin{itemize}
        \item This makes $\mathcal{C}(X)$ into a vector space.
        \item We can now define our metric $d(f,g)$ by $d(f,g)=\norm{f-g}$.
    \end{itemize}
    \item $f_n\to f$ $\Longleftrightarrow$ $f$ is bounded.
    \begin{itemize}
        \item $f_n\to f$ uniformly $\Longleftrightarrow$ $\lim_{n\to\infty}\sup|f_n(x)-f(x)|=0$ $\Longleftrightarrow$ $f_n\to f$ is $\mathcal{C}(X)$.
    \end{itemize}
    \item Corollary to the Weierstra{\ss} M-test: $\mathcal{C}(X)$ is complete (i.e., all uniformly Cauchy sequences converge).
    \begin{itemize}
        \item Assume $\{f_n\}$ is Cauchy. Then by the Cauchy criterion for uniform convergence, $f_n$ converges uniformly to some $f$. But this $f$ must be continuous, too, meaning $f\in\mathcal{C}(X)$.
    \end{itemize}
    \item \marginnote{2/4:}Plan.
    \begin{enumerate}
        \item $\int\lim f_n=\lim\int f_n$.
        \item $\dd{x}\lim f_n=\lim\dd{x}f_n$.
        \item Definitions: Pointwise/uniform boundedness, equicontinuity.
    \end{enumerate}
    \item Theorem: $f_n:[a,b]\to\R$ integrable and $f_n\to f$ uniformly implies $f$ is integrable and
    \begin{equation*}
        \int_a^bf = \lim_{n\to\infty}\int_a^bf_n
    \end{equation*}
    \begin{itemize}
        \item Plan:
        \begin{enumerate}
            \item Show $f$ is integrable.
            \item Show $\int f=\lim\int f_n$.
        \end{enumerate}
        \item Proof:
        \begin{itemize}
            % \item Fix some partition $P$ and write
            % \begin{align*}
            %     U(f_n,P) &= \sum_{i=1}^kM_i^n\Delta x_i&
            %         L(f_n,P) &= \sum_{i=1}^km_i^n\Delta x_i\\
            %     U(f,P) &= \sum_{i=1}^kM_i\Delta x_i&
            %         L(f,P) &= \sum_{i=1}^km_i\Delta x_i
            % \end{align*}
            % \item WTS: For large enough $n$, we can make all $|M_i^n-M_i|<\epsilon$ and $|m_i^n-m_i|<\epsilon$.
            % \item We can do the above by uniform convergence.
            % \item We end up with
            % \begin{align*}
            %     |U(f_n,P)-U(f,P)| &\leq \sum_{i=1}^n|M_i^n-M_i|\Delta x_i\\
            %     &< \epsilon(b-a)
            % \end{align*}
            \item Let $\epsilon_n=\sup_{x\in[a,b]}|f(x)-f_n(x)|$.
            \item Since $f_n\to f$ uniformly, $\epsilon_n\to 0$ as $n\to\infty$.
            \item By definition, $f_n-\epsilon_n\leq f\leq f_n+\epsilon_n$.
            \item Thus, by Theorems 6.4 and 6.5,
            \begin{equation*}
                \int_a^b(f_n-\epsilon_n) = \underaccent{\bar}{\int}(f_n-\epsilon_n)
                \leq \underaccent{\bar}{\int}f
                \leq \bar{f}
                \leq \int_a^b(f_n+\epsilon_n)
            \end{equation*}
            \item It follows since
            \begin{equation*}
                0 \leq \bar{\int}f-\underaccent{\bar}{\int}f
                \leq \int_a^b(f_n+\epsilon_n)-\int_a^b(f_n-\epsilon_n)
                = (b-a)...
            \end{equation*}
            that $f$ is integrable.
            \item Hence,
            \begin{align*}
                \int_a^b(f_n-\epsilon_n) &\leq \int_a^bf \leq \int_a^b(f_n-\epsilon_n)\\
                \left| \int_a^bf_n-\int_a^bf \right| &\leq \epsilon_n\\
                \lim_{n\to\infty}\int_a^bf_n = \int_a^bf
            \end{align*}
        \end{itemize}
    \end{itemize}
    \item Theorem: $f_n:[a,b]\to\R$, each $f_n$ differentiable, $f_n\to f$ pointwise, and $(f_n)'\to g$ uniformly implies that $f$ is differentiable and $f'=g$.
    \begin{itemize}
        \item Note that you can do better: Substituting $f_n(x_0)$ converging for some $x_0\in[a,b]$ for $f_n\to f$ pointwise still implies the desired result.
        \item Idea: We use the $\epsilon/3$ trick; $2/3$ will be easy and $1/3$ will be tricky.
        \item Goal: We want
        \begin{equation*}
            \left| \frac{f(x)-f(x_0)}{x-x_0}-g(x_0) \right| < \epsilon
        \end{equation*}
        for some $\delta$ with $0<|x-x_0|<\delta$. We will show that
        \begin{align*}
            {}& \left| \frac{f(x)-f(x_0)}{x-x_0}-\frac{f_N(x)-f_N(x_0)}{x-x_0}+\frac{f_N(x)-f_N(x_0)}{x-x_0}-f'_N(x_0)+f'_N(x_0)-g(x_0) \right|\\
            \leq{}& \left| \frac{f(x)-f(x_0)}{x-x_0}-\frac{f_N(x)-f_N(x_0)}{x-x_0} \right|+\left| \frac{f_N(x)-f_N(x_0)}{x-x_0}-f'_N(x_0) \right|+\left| f'_N(x_0)-g(x_0) \right|
        \end{align*}
        \begin{itemize}
            \item For the middle inequality, use Chapter 5, Exercise 8.
            \item For the right inequality, use the uniform convergence condition.
            \item For the left inequality, it will suffice to show the Cauchy condition
            \begin{equation*}
                \left| \frac{f_n(x)-f_n(x_0)}{x-x_0}-\frac{f_m(x)-f_m(x_0)}{x-x_0} \right| < \frac{\epsilon}{3}
            \end{equation*}
            so, noting that the left term above is equal to
            \begin{equation*}
                \left| \frac{[f_n(x)-f_m(x)]-[f_n(x_0)-f_m(x_0)]}{x-x_0} \right|
            \end{equation*}
            which is equal to $|f_n'(c)-f_m'(c)|$ by the MVT, from which we can apply the Cauchy form of the uniform convergence of $(f_n)'$ condition.
        \end{itemize}
    \end{itemize}
    \item \textbf{Pointwise bounded} ($\{f_n\}$): A sequence of real functions $\{f_n\}$ such that for all $x\in X$, there exists $M_x\in\R$ such that $|f_n(x)|\leq M_x$ for all $n\in\N$.
    \item \textbf{Uniformly bounded} ($\{f_n\}$): A sequence of real functions $\{f_n\}$ for which there exists $M\in\R$ such that for all $x\in X$ and $n\in\N$, $|f_n(x)|\leq M$.
    \item Proposition: $f_n:E\to\R$, $\{f_n\}$ is pointwise bounded, and $E$ is countable implies there is a subsequence $\{f_{n_k}\}$ that converges pointwise.
    \begin{itemize}
        \item Enumerate $E=\{x_1,x_2,\dots\}$.
        \item Then since $\{f_n(x_m)\}$ is bounded for all $m$ by hypothesis, it always has a convergent subsequence.
        \item The claim is if you look at the sequence of diagonal functions, it is such a subsequence, i.e., if $f_1(x_1)$ is the first term for $x_1$, $f_3(x_2)$ is the second term for $x_2$, $f_{11}(x_3)$ is the third term for $x_3$, and so on, $f_1,f_3,f_{11},\dots$ is such a subsequence.
    \end{itemize}
\end{itemize}




\end{document}