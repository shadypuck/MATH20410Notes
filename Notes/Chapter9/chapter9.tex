\documentclass[../notes.tex]{subfiles}

\pagestyle{main}
\renewcommand{\chaptermark}[1]{\markboth{\chaptername\ \thechapter\ (#1)}{}}
\setcounter{chapter}{8}

\begin{document}




\chapter{Functions of Several Variables}
\section{Notes}
\begin{itemize}
    \item \marginnote{2/14:}Plan:
    \begin{enumerate}
        \item Warm-up with matrices.
        \item The total derivatives of $f:\R^n\to\R^m$ ($n=m=2$, i.e., $f:\C\to\C$).
        \item Basic properties: Chain rule, relation with partial derivatives, implicit function theorem.
    \end{enumerate}
    \item Let $V,W$ be finite-dimensional vector spaces over $\R$. We let $L(V,W)$ be the vector space of all linear transformations $\phi:V\to W$.
    \item If we pick bases $N_1,\dots,N_n$ of $V$ and $w_1,\dots,w_m$ of $W$, then $V\cong\R^n$ and $W\cong\R^m$. It follows that $L(V,W)\cong\R^{mn}$.
    \item $L(V,W)\times L(W,U)\xrightarrow{\text{compose}}L(V,U)$, i.e., $\R^{mn}\times\R^{nl}\xrightarrow[\text{mult.}]{\text{matrix}}\R^{ml}$.
    \item Sup norm: If $A$ is an $m\times n$ real matrix, then $\norm{A}=\sup_{\substack{\vec{x}\in\R^n\\|\vec{x}|=1}}|A\vec{x}|$.
    \begin{itemize}
        \item Basic properties:
        \begin{enumerate}
            \item $|A\vec{x}|\leq\norm{A}|x|$.
            \item $\norm{A}<\infty$ and all $A:\R^n\to\R^m$ are uniformly continuous.
            \item $\norm{A}=0\Longleftrightarrow A=0$.
            \item $\norm{cA}=|c|\norm{A}$.
            \item $\norm{A+B}\leq\norm{A}+\norm{B}$.
            \item $\norm{AB}\leq\norm{A}\norm{B}$.
        \end{enumerate}
        \item Note that we get a metric space structure on $L(V,W)$ by defining $d(A,B)=\norm{A-B}$.
    \end{itemize}
    \item Proves that 1 and 2 imply the uniform continuity of all $A$ (via Lipschitz continuity).
    \item \textbf{Differentiable} (multivariate function $f$ at $\vec{x}_0$): A function $f:U\to\R^m$ ($U\subset\R^n$) such that to $\vec{x}_0\in U$ there corresponds some linear transformation $A:\R^n\to\R^m$ such that
    \begin{equation*}
        \lim_{\vec{h}\to\bm{0}}\frac{|f(\vec{x}_0-\vec{h})-f(\vec{x}_0)-A\vec{h}}{|\vec{h}|} = 0
    \end{equation*}
    \item \textbf{Total derivative} (of $f$ multivariate at $\vec{x}_0$): The linear transformation $A$ in the above definition. \emph{Denoted by} $\bm{f'(\vec{x}_0)}$.
    \item "An proof and progress in mathematics" - Thurston.
    \begin{itemize}
        \item Relating to the old one dimensional derivative.
        \item A paper we'd find rather impressionistic right now.
    \end{itemize}
    \item Propositions ahead of us.
    \begin{itemize}
        \item Proposition: Suppose that $f$ is differentiable at $\vec{x}_0\in U$ and $A,B$ are both derivatives of $f$ at $\vec{x}_0$. Then $A=B$.
        \item Proposition: Differentiable implies continuous.
        \item Proposition: Sum rule, product rule, quotient rule.
    \end{itemize}
\end{itemize}




\end{document}